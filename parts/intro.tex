
\section{Introduction}

%The study and comprehension of biological systems relies on our ability to build models of such systems and
%study their properties. The way a molecular system behaves can be represented as large interaction graphs 
%composed of genes, proteins and complexes which interact to produce different dynamics in the system. 
%Thanks to the availability of gene time-series expression data, we can now observe and characterize the 
%dynamics of such components in the context of biological regulatory networks.


%These refinements were: \emph{(i)} modeling the expression of the E-cadherin signal as a discrete pulse,
%\emph{(ii)} imposing a weak synchronization of multiple regulators of a molecule, and \emph{(iii)} modeling 
%the decay of the strength of a transcription factor over the regulated gene.



% no \IEEEPARstart
The comprehension of the mechanisms involved in the regulation of a cell-based biological system is a fundamental 
issue. These mechanisms can be modeled as biological regulatory networks, which analysis requires to preliminary build a 
mathematical or computational model. 
By just considering qualitative regulatory effects between components, biologic regulatory networks
depict fairly well biological systems, and can be built upon public repositories such as the Pathways 
Interaction Database \cite{schaefer2009pid}, and 
hiPathDB \cite{yu2012hipathdb} for human regulatory knowledge.

%the aims of the work (1first sentence & the background)

This work aims to propose a dynamical model of large-scale systems based on the formal integration 
(complete validation/invalidation) of high-throughput experimental time-series data.  So far this 
idea has been addressed separately by approaches that either: (a) focus first on
modeling at small-scale the system and then on refining or improving it through the fitting with
some data points, such as methods based on differential equations \cite{tyson2003sniffers,batt2005validation,mobashir2012simulated}, 
(b) integrate in an efficient and complete fashion large-scale models
and high-throughput data regardless from the system dynamics \cite{guziolowski2013exhaustively,mitsos2009identifying}, or 
(c) fit dynamical data to middle-scale networks using an sampling of the space of behaviors stochastic approaches, and
therefore without guarantee on finding global optima \cite{macnamara2012state}. Therefore, with this
work we intend to fill the gaps between the previously cited methodologies and converge to a more
realistic model of biological behavior.


%proposer une petite discussion sur les autres approches formelles de modélisation


%le choix que nous avons fais. Il faudra  détailler les avantages. Dire pourquoi nous choisissons le PH et pas un aute formalisme
%les grandes lignes de cette justification sont les suivantes: formalisme très expressif pour les BRN, simulation concurente 
% provenant de l'algèbre des processus, abstraction permettant de faire de analyse statiques de certaines propriétés. 

For modeling and analyzing stochastic and concurent biological systems, many formalisms have been introduced,such as 
stochastic Petri Nets which is a suitable approach suitable for the representation of parallel systems \cite{molloy1982performance}. 
They have been successfully and systematically applied in many areas, and the specification Petri Nets
allows an accurate modeling of a wide range of systems including biological systems \cite{heiner2008petri}. The major 
problem of Stochastic Petri Nets is that they do not generally lead to compact models. In addition,
they do not provide results to deal with the state space explosion and are thus generally computationally
expensive. Another approach is Stochastic pi-calculus introduced by \cite{priami1995stochastic} and used in 
\cite{maurin2009modeling} for the modeling of biological systems. Stochastic pi-calculus have a good
expressivity and are well adapted for the use of compositionnal approach.

%TODO
For modeling and analyzing the biological system we rely on a stochastic pi-calculus formalism which is the Process Hitting (PH) framework \cite{PMR10-TCSB}, 
since it is especially useful for studying systems composed of biochemical interactions, and provides
stochastic simulation as well as efficient static methods to model dynamical properties of the system.
The PH framework uses qualitative and discrete information of the system, without requiring enormous parameter estimation tasks
 for its stochastic simulation. 
So far, this method has been successfully demonstrated only on very well-known systems and without exploiting 
high-throughput measures. We believe, however, that the use of high-throughput data has become unavoidable with 
the advent of massive, publicly available data sets in the form of well-standardized DNA microarray data and, 
more recently, in the form of phospho-proteomics data.  




% les contributions du travail que nous présentons: Génération automatique des modèles en PH(avec et 
% sans synchronisation), Estiamtion des paramètres des times series data Et intégration des paramètres 
% dans le modèle.
%TODO
The main  results of this work are: (i) automatic PH generation                                                         
from a biological system composed of biochemical reactions, and extracted from public databases;
(ii) discretization approach of time-series expression data, so we can 
reproduce these traces by using in a first attempt the PH stochastic simulation, and afterwards 
perform static analysis methods for reachability analyses to satisfy this data; and (iii) estimation of the the temporal and stochastic
parameters of the simulation, based on statistical analyses of the full-compendium of time-series expression data.
The biological system used as a case-study for this work is a cell-based model of skin differentiation, which 
is of key importance in  wound healing. 

% You must have at least 2 lines in the paragraph with the drop letter
% (should never be an issue)


%\hfill mds
 
\hfill March 20, 2015